\section{Config file}
For optional parameters you can use the --configuration flag to store some standard parameters in a config file. The --configuration flag has one parameter, the path to the config file.
The command line takes precedence if arguments are defined in both.

\subsection{File format}
To add options to the config you start with the command (consens or coverage) you want to add the flags and then on every new line a flag with arguments can be added as seen in figure~\ref{fig:config-file}. You can use '\#' to mark a line as a comment.
\\
\begin{figure}[h]
\caption{config.txt}
\fbox{%
    \parbox{\textwidth}{%
\# The consensus command.\\
consensus\\
-r 100-200\\
-ea 0.2-08\\
\# The coverage command\\
coverage\\
-r 100-200
}
}
\label{fig:config-file}
\end{figure}

You can now execute a command with: \\
\begin{markdown}
'consensus -c variants.vcf --config config.txt'
\end{markdown}
\\
This will execute the command that will have the same result as the command:
\\
\begin{markdown}
'consensus -c variants.vcf -r 100-200 -ea 0.2-0.8'
\end{markdown}
\\
\\
If both the option is defined in the command-line and in the config file the command-line argument will overwrite the config-argument. For example:\\ 
\begin{markdown}
'consensus -c variants.vcf --config config.txt -r 200-300'
\end{markdown}
\\
Will have the same result as:
\\
\begin{markdown}
'consensus -c variants.vcf -r 200-300 -ea 0.2-0.8'
\end{markdown}